\addcontentsline{toc}{chapter}{Abstract}
\chapter*{Abstract}
For nearly 60 years, heat stored within the earth has been used to generate emission-free, baseload electricity, making up a large percentage of energy production in many parts of the world. Monitoring these resources requires a large number of complementary datasets collected from wells drilled into deep geothermal reservoirs or from instrumentation deployed at the surface. In this thesis, we focus on a catalog of microearthquakes recorded at two New Zealand geothermal fields, Ngatamariki and Rotokawa, for the years 2012--2015 and use these data to improve the knowledge of reservoir behavior.

Seismicity occurs within geothermal reservoirs as a result of changes in pressure, temperature and stress, which induce shear failure on preexisting faults and fractures. These microearthquakes occur frequently, often every few seconds, and therefore provide a tool for assessing reservoir properties with dense spatial and temporal resolution. However, operators often rely on automated seismic monitoring that provides basic, imprecise locations and therefore limited helpful information to reservoir engineers. Here we undertake a more detailed analysis of an induced seismic sequence, incorporating improved detection, precise relocation, magnitude estimation and focal mechanism determination, which yields significantly more useful information to geothermal operators.

We begin with an automatically-processed seismic catalog for the Rotokawa and Ngatamariki geothermal fields in New Zealand, provided by GNS Science. We then expand the number of events in this catalog two-fold using a matched-filter detection technique ($\sim$9000 events total), while decreasing the location uncertainties by roughly an order of magnitude using double-difference relocation methods. We calculate the focal mechanisms solutions for 982 of these events and use them to invert for the stress state over space and time within the reservoirs.

At Ngatamariki, these results constitute the first detailed analysis of seismicity at a newly-developed resource.  As is often observed in other settings worldwide, seismic events at Ngatamariki occur in areas of fluid injection, but rarely in areas of extraction. Variations in seismicity between the two injection zones (referred to here as `northern' and `southern') underscore the geological and structural differences between these parts of the field. In the north, a higher $b$-value (1.84) and unique stress state with no vertical axis of principle stress indicate that the emplacement of an intrusive body may have created a  dense fracture network and deviated stress state in that part of the reservoir. In contrast, lower $b$-value (1.2) and a normal stress regime (NE-SW S$_{Hmax}$) in the south, where injection occurs into a large fault zone, are similar to regional observations. In addition, we observe that stimulation (i.e. permeability increase) of injection wells prior to plant startup occurs either in the absence of, or is poorly correlated with, induced seismicity. This suggests that hydro-shear, the process of inducing seismicity through increased pore pressure at critically-stressed fractures, is not the dominant mechanism of permeability increase at geothermal wells. Instead, we infer aseismic processes, likely thermally-induced fracture opening, to dominate well stimulation in geothermal settings.

Our seismic catalog for Rotokawa resembles the catalog compiled for the four years (2008--2012) prior to our study period. We identify the same seismically-active area of the reservoir as was identified previously and also infer the lack of seismicity in the production field to be partially-explained by NE-SW-striking faults, which act as barriers to fluid flow. However, the large number of events and high-precision locations in our catalog allow us to refine the orientation of one previously-modeled fault and define the location for another using earthquake hypocenters. Although it has been proposed that power plant shutdown and startup coincide with times of increased seismic activity, we do not find such a correlation here.

Areas of high $b$-value at Rotokawa may correspond to areas of high pore-fluid pressure, which allows failure on non-critically-stressed faults, or to areas of reservoir cooling where tensional failure may create a densely-fractured zone. Finally, stress inversion results show a normal faulting regime throughout the reservoir. However, in areas near the injection wells, a lower stress ratio and anti-clockwise rotation of S$_{Hmax}$ may indicate the influence of reservoir cooling on the stress state. Temporal analysis of the stress ratio and S$_{Hmax}$ for the focal mechanisms nearest the injection wells shows that stress ratio decreases dramatically with the introduction of injection at well RK23, before gradually increasing to pre-injection levels. We propose a scenario wherein the shape of the cooled zone dictates which principle stress axes is most reduced by cooling, thereby affecting the stress ratio.

The work contained herein contributes a new, combined four-year earthquake catalog for the Ngatamariki and Rotokawa geothermal fields in the Central Taup\={o} Volcanic Zone of New Zealand. This catalog spans the developmental and startup period for the Ngatamariki power plant and a period of equilibration for the Rotokawa field following the startup of the Nga Awa Purua plant in 2010. This catalog helps constrain field structure, and indicates where fluid is flowing within the reservoir. It also offers clues as to the nature of the fracture network in the spaces away from wells. Stress inversions of earthquake focal mechanisms also constrain which fractures are likely to be open in a given volume of the reservoir. This catalog significantly updates and corroborates the current understanding of the Ngatamariki and Rotokawa reservoirs and should prove useful to the reservoir engineers and modelers.

